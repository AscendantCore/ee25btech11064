\let\negmedspace\undefined
\let\negthickspace\undefined
\documentclass[journal]{IEEEtran}
\usepackage[a4paper, margin=10mm, onecolumn]{geometry}
\usepackage{lmodern} % Ensure lmodern is loaded for pdflatex
\usepackage{tfrupee} % Include tfrupee package

\setlength{\headheight}{1cm} % Set the height of the header box
\setlength{\headsep}{0mm}  % Set the distance between the header box and the top of the text

\usepackage{gvv-book}
\usepackage{gvv}
\usepackage{cite}
\usepackage{amsmath,amssymb,amsfonts,amsthm}
\usepackage{algorithmic}
\usepackage{graphicx}
\usepackage{float}
\usepackage{textcomp}
\usepackage{xcolor}
\usepackage{txfonts}
\usepackage{listings}
\usepackage{enumitem}
\usepackage{mathtools}
\usepackage{gensymb}
\usepackage{comment}
\usepackage[breaklinks=true]{hyperref}
\usepackage{tkz-euclide} 
\usepackage{listings}
% \usepackage{gvv}                                        
\def\inputGnumericTable{}                                 
\usepackage[latin1]{inputenc}                                
\usepackage{color}                                            
\usepackage{array}                                            
\usepackage{longtable}                                       
\usepackage{calc}                                             
\usepackage{multirow}                                         
\usepackage{hhline}                                           
\usepackage{ifthen}                                           
\usepackage{lscape}
\usepackage{tikz}
\usetikzlibrary{patterns}

\begin{document}

\bibliographystyle{IEEEtran}
\vspace{3cm}

\title{5.10.2}
\author{EE25BTECH11064 - Yojit Manral}

\maketitle
% \maketitle
% \newpage
% \bigskip
{\let\newpage\relax\maketitle}
\renewcommand{\thefigure}{\theenumi}
\renewcommand{\thetable}{\theenumi}
\setlength{\intextsep}{10pt} % Space between text and float

\textbf{Question:}\\
Balance the following chemical equation:
\begin{align}
    NaOH + H_2SO_4 \rightarrow Na_2SO_4 + H_2O
\end{align}

\textbf{Solution:}\\
$\rightarrow$ Let the balanced version of (1) be
\begin{align}
    x_1NaOH + x_2H_2SO_4 \rightarrow x_3Na_2SO_4 + x_4H_2O
\end{align}
$\rightarrow$ This results in the following equations
\begin{align}
    (x_1 - 2x_3)Na &= 0 \\
    (x_1 + 4x_2 - 4x_3 - x_4)O &= 0 \\
    (x_1 + 2x_2 - 2x_4)H &= 0 \\
    (x_2 - x_3)S &= 0
\end{align}
$\rightarrow$ Which can further be expressed as
\begin{align}
    (1x_1 + 0x_2 - 2x_3 + 0x_4)Na &= 0 \\
    (1x_1 + 4x_2 - 4x_3 - 1x_4)O &= 0 \\
    (1x_1 + 2x_2 + 0x_3 - 2x_4)H &= 0 \\
    (0x_1 + 1x_2 - 1x_3 + 0x_4)S &= 0
\end{align}
$\rightarrow$ Giving us the matrix equation
\begin{align}
    \myvec{1&0&-2&0\\1&4&-4&-1\\1&2&0&-2\\0&1&-1&0}\vec{x} = 0, \hspace{0.5cm} \vec{x} = \myvec{x_1\\x_2\\x_3\\x_4}
\end{align}
$\rightarrow$ Now, (11) can be reduced as follows
\begin{align}
&\myvec{1&0&-2&0\\1&4&-4&-1\\1&2&0&-2\\0&1&-1&0}\xrightarrow[R_3 \leftrightarrow R_3 - R_1]{R_2 \leftrightarrow R_2 - R_1}\myvec{1&0&-2&0\\0&4&-2&-1\\0&2&2&-2\\0&1&-1&0} \\
\xrightarrow{R_2 \leftrightarrow (1/4)R_2}&\myvec{1&0&-2&0\\0&1&-1/2&-1/4\\0&2&2&-2\\0&1&-1&0}\xrightarrow[R_4 \leftrightarrow R_4 - R_2]{R_3 \leftrightarrow R_3 - 2R_2}\myvec{1&0&-2&0\\0&1&-1/2&-1/4\\0&0&3&-3/2\\0&0&-1/2&1/4} \\
\xrightarrow[R_1 \leftrightarrow R_1 + 2R_3]{R_3 \leftrightarrow (1/3)R_3}&\myvec{1&0&0&-1\\0&1&-1/2&-1/4\\0&0&1&-1/2\\0&0&-1/2&1/4}\xrightarrow[R_4 \leftrightarrow R_4 + (1/2)R_3]{R_2 \leftrightarrow R_2 + (1/2)R_3}\myvec{1&0&0&-1\\0&1&0&-1/2\\0&0&1&-1/2\\0&0&0&0}
\end{align}
$\rightarrow$ Thus
\begin{align}
    x_1 = x_4, x_2 = \frac{1}{2}x_4, x_3 = \frac{1}{2}x_4 \\
    \implies \vec{x} = x_4\myvec{1\\1/2\\1/2\\1} = \myvec{2\\1\\1\\2}
\end{align}
\hspace{0.3cm} by substituting $x_4 = 2$. Hence, (2) finally becomes
\begin{align}
    2NaOH + H_2SO_4 \rightarrow Na_2SO_4 + 2H_2O
\end{align}
\end{document}
