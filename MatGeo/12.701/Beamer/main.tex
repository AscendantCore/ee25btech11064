\documentclass{beamer}
\mode<presentation>
\usepackage{amsmath,amssymb,mathtools}
\usepackage{textcomp}
\usepackage{gensymb}
\usepackage{adjustbox}
\usepackage{subcaption}
\usepackage{enumitem}
\usepackage{multicol}
\usepackage{listings}
\usepackage{url}
\usepackage{graphicx} % <-- needed for images
\def\UrlBreaks{\do\/\do-}

\usetheme{Boadilla}
\usecolortheme{lily}
\setbeamertemplate{footline}{
  \leavevmode%
  \hbox{%
  \begin{beamercolorbox}[wd=\paperwidth,ht=2ex,dp=1ex,right]{author in head/foot}%
    \insertframenumber{} / \inserttotalframenumber\hspace*{2ex}
  \end{beamercolorbox}}%
  \vskip0pt%
}
\setbeamertemplate{navigation symbols}{}

\lstset{
  frame=single,
  breaklines=true,
  columns=fullflexible,
  basicstyle=\ttfamily\tiny   % tiny font so code fits
}

\numberwithin{equation}{section}

% ---- your macros ----
\providecommand{\nCr}[2]{\,^{#1}C_{#2}}
\providecommand{\nPr}[2]{\,^{#1}P_{#2}}
\providecommand{\mbf}{\mathbf}
\providecommand{\pr}[1]{\ensuremath{\Pr\left(#1\right)}}
\providecommand{\qfunc}[1]{\ensuremath{Q\left(#1\right)}}
\providecommand{\sbrak}[1]{\ensuremath{{}\left[#1\right]}}
\providecommand{\lsbrak}[1]{\ensuremath{{}\left[#1\right.}}
\providecommand{\rsbrak}[1]{\ensuremath{\left.#1\right]}}
\providecommand{\brak}[1]{\ensuremath{\left(#1\right)}}
\providecommand{\lbrak}[1]{\ensuremath{\left(#1\right.}}
\providecommand{\rbrak}[1]{\ensuremath{\left.#1\right)}}
\providecommand{\cbrak}[1]{\ensuremath{\left\{#1\right\}}}
\providecommand{\lcbrak}[1]{\ensuremath{\left\{#1\right.}}
\providecommand{\rcbrak}[1]{\ensuremath{\left.#1\right\}}}
\theoremstyle{remark}
\newtheorem{rem}{Remark}
\newcommand{\sgn}{\mathop{\mathrm{sgn}}}
\providecommand{\abs}[1]{\left\vert#1\right\vert}
\providecommand{\res}[1]{\Res\displaylimits_{#1}}
\providecommand{\norm}[1]{\lVert#1\rVert}
\providecommand{\mtx}[1]{\mathbf{#1}}
\providecommand{\mean}[1]{E\left[ #1 \right]}
\providecommand{\fourier}{\overset{\mathcal{F}}{ \rightleftharpoons}}
\providecommand{\system}{\overset{\mathcal{H}}{ \longleftrightarrow}}
\providecommand{\dec}[2]{\ensuremath{\overset{#1}{\underset{#2}{\gtrless}}}}
\newcommand{\myvec}[1]{\ensuremath{\begin{pmatrix}#1\end{pmatrix}}}
\let\vec\mathbf

\title{MatGeo Presentation - Problem 12.701}
\author{EE25BTECH11064 - Yojit Manral}
\date{}

\begin{document}

\frame{\titlepage}
\begin{frame}{Question}
Let $ (\cdot, \cdot) : R^n \times R^n \rightarrow R $ be the inner product. Consider
\begin{align*}
    P &: |(\vec{u}, \vec{v})| \leq \frac{(\vec{u}, \vec{u})+(\vec{v}, \vec{v})}{2} \forall \vec{u}, \vec{v}. \\
    Q &: \text{If }(\vec{u}, \vec{v}) = (2\vec{u}, \vec{v}) \hspace{0.2cm} \forall \vec{v} \text{, then }\vec{u} = 0.
\end{align*}
Then
\begin{enumerate}[label=(\alph*)]
\begin{multicols}{2}
    \item both P, Q are true
    \item P is true, Q is false
    \item P is false, Q is true
    \item both P, Q are false
\end{multicols}
\end{enumerate}
\end{frame}

\begin{frame}{Solution}
$\rightarrow$ We can take the inner product as the dot product, since all inner products have the same properties as satisfied by the dot product.
\begin{align}
    \implies (\vec{u}, \vec{v}) = \vec{u}^T\vec{v}
\end{align}
$\rightarrow$ For Statement P
\begin{align}
    |(\vec{u}, \vec{v})| &\leq \norm{\vec{u}}\norm{\vec{v}} && \text{(Cauchy-Schwarz Inequality)} \\
    \frac{(\vec{u}, \vec{u}) + (\vec{v}, \vec{v})}{2} &= \frac{\norm{\vec{u}}^2 + \norm{\vec{v}}^2}{2} \\
    \norm{\vec{u}}\norm{\vec{v}} &\leq \frac{\norm{\vec{u}}^2 + \norm{\vec{v}}^2}{2} && \text{(G.M. $\leq$ A.M.)} \\
    \implies |(\vec{u}, \vec{v})| &= \frac{(\vec{u}, \vec{u})+(\vec{v}, \vec{v})}{2} && \implies \textit{P is true}
\end{align}
\end{frame}

\begin{frame}{Solution}
$\rightarrow$ For Statement Q
\begin{align}
    (\vec{u}, \vec{v}) &= (2\vec{u}, \vec{v}) && (\forall \vec{v} \in R^n) \\
    (\vec{u}, \vec{v}) - (2\vec{u}, \vec{v}) &= 0 \\
    (\vec{u} - 2\vec{u}, \vec{v}) &= 0 \\
    (-\vec{u}, \vec{v}) &= 0 && (\forall \vec{v} \in R^n) \\
    \implies \vec{u} &= 0 && \implies \textit{Q is true}
\end{align}

$\longrightarrow$ Therefore, \textit{(b) both P, Q are true} is the correct option.
\end{frame}

\end{document}
