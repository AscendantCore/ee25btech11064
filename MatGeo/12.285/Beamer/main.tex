\documentclass{beamer}
\mode<presentation>
\usepackage{amsmath,amssymb,mathtools}
\usepackage{textcomp}
\usepackage{gensymb}
\usepackage{adjustbox}
\usepackage{subcaption}
\usepackage{enumitem}
\usepackage{multicol}
\usepackage{listings}
\usepackage{url}
\usepackage{graphicx} % <-- needed for images
\def\UrlBreaks{\do\/\do-}

\usetheme{Boadilla}
\usecolortheme{lily}
\setbeamertemplate{footline}{
  \leavevmode%
  \hbox{%
  \begin{beamercolorbox}[wd=\paperwidth,ht=2ex,dp=1ex,right]{author in head/foot}%
    \insertframenumber{} / \inserttotalframenumber\hspace*{2ex}
  \end{beamercolorbox}}%
  \vskip0pt%
}
\setbeamertemplate{navigation symbols}{}

\lstset{
  frame=single,
  breaklines=true,
  columns=fullflexible,
  basicstyle=\ttfamily\tiny   % tiny font so code fits
}

\numberwithin{equation}{section}

% ---- your macros ----
\providecommand{\nCr}[2]{\,^{#1}C_{#2}}
\providecommand{\nPr}[2]{\,^{#1}P_{#2}}
\providecommand{\mbf}{\mathbf}
\providecommand{\pr}[1]{\ensuremath{\Pr\left(#1\right)}}
\providecommand{\qfunc}[1]{\ensuremath{Q\left(#1\right)}}
\providecommand{\sbrak}[1]{\ensuremath{{}\left[#1\right]}}
\providecommand{\lsbrak}[1]{\ensuremath{{}\left[#1\right.}}
\providecommand{\rsbrak}[1]{\ensuremath{\left.#1\right]}}
\providecommand{\brak}[1]{\ensuremath{\left(#1\right)}}
\providecommand{\lbrak}[1]{\ensuremath{\left(#1\right.}}
\providecommand{\rbrak}[1]{\ensuremath{\left.#1\right)}}
\providecommand{\cbrak}[1]{\ensuremath{\left\{#1\right\}}}
\providecommand{\lcbrak}[1]{\ensuremath{\left\{#1\right.}}
\providecommand{\rcbrak}[1]{\ensuremath{\left.#1\right\}}}
\theoremstyle{remark}
\newtheorem{rem}{Remark}
\newcommand{\sgn}{\mathop{\mathrm{sgn}}}
\providecommand{\abs}[1]{\left\vert#1\right\vert}
\providecommand{\res}[1]{\Res\displaylimits_{#1}}
\providecommand{\norm}[1]{\lVert#1\rVert}
\providecommand{\mtx}[1]{\mathbf{#1}}
\providecommand{\mean}[1]{E\left[ #1 \right]}
\providecommand{\fourier}{\overset{\mathcal{F}}{ \rightleftharpoons}}
\providecommand{\system}{\overset{\mathcal{H}}{ \longleftrightarrow}}
\providecommand{\dec}[2]{\ensuremath{\overset{#1}{\underset{#2}{\gtrless}}}}
\newcommand{\myvec}[1]{\ensuremath{\begin{pmatrix}#1\end{pmatrix}}}
\let\vec\mathbf

\title{MatGeo Presentation - Problem 12.285}
\author{EE25BTECH11064 - Yojit Manral}
\date{}

\begin{document}

\frame{\titlepage}
\begin{frame}{Question}
Let $\alpha = e^{2\pi i/5}$ and the matrix
\begin{align}
    \vec{M} = \myvec{1&\alpha&\alpha^2&\alpha^3&\alpha^4\\0&\alpha&\alpha^2&\alpha^3&\alpha^4\\0&0&\alpha^2&\alpha^3&\alpha^4\\0&0&0&\alpha^3&\alpha^4\\0&0&0&0&\alpha^4}
\end{align}
Then the trace of the matrix $\vec{I} + \vec{M} + \vec{M}^2$ is
\begin{enumerate}[label=(\alph*)]
\begin{multicols}{4}
    \item $5$
    \item $0$
    \item $3$
    \item $-5$
\end{multicols}
\end{enumerate}
\end{frame}

\begin{frame}{Solution}
$\rightarrow$ As $\alpha$ is the fifth root of unity, we have
\begin{align}
    \alpha^5 - 1 = 0 \implies 1 + \alpha + \alpha^2 + \alpha^3 + \alpha^4 = 0
\end{align}
$\rightarrow$ According to the properties of trace
\begin{align}
    trace(\vec{A}_{n\times n}) &\triangleq \sum_{i=1}^{n} a_{ii} \\
    trace(\vec{I}+\vec{M}+\vec{M}^2) &= trace(\vec{I})+trace(\vec{M})+trace(\vec{M}^2) \\
    trace(\vec{I}_{5\times5}) &= \sum_{i=1}^{5} 1 = 5
\end{align}
$\rightarrow$ From (1) and (3), we get
\begin{align}
    trace(\vec{M}) = \sum_{i=1}^{5} m_{ii} = 1 + \alpha + \alpha^2 + \alpha^3 + \alpha^4
\end{align}
\end{frame}

\begin{frame}{Solution}
$\rightarrow$ Since $\vec{M}$ is an upper triangular matrix
\begin{align}
    trace(\vec{M}^2) = \sum_{i=1}^{5} m_{ii}^2 = (1)^2 + (\alpha)^2 + (\alpha^2)^2 + (\alpha^3)^2 + (\alpha^4)^2
\end{align}
$\rightarrow$ On adding (5), (6), and (7), using (2) and (4), we get
\begin{align}
    trace(\vec{I}+\vec{M}+\vec{M}^2) = 5 &+ (1 + \alpha + \alpha^2 + \alpha^3 + \alpha^4) \notag \\ &+ (1 + \alpha^2 + \alpha^4 + \alpha\alpha^5 + \alpha^3\alpha^5) \\
    = 5 &+ (0) + (0) = 5
\end{align}
$\rightarrow$ Therefore, (a) 5 is the correct option.
\end{frame}

\end{document}
