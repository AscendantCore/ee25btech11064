\let\negmedspace\undefined
\let\negthickspace\undefined
\documentclass[journal]{IEEEtran}
\usepackage[a4paper, margin=10mm, onecolumn]{geometry}
\usepackage{lmodern} % Ensure lmodern is loaded for pdflatex
\usepackage{tfrupee} % Include tfrupee package

\setlength{\headheight}{1cm} % Set the height of the header box
\setlength{\headsep}{0mm}  % Set the distance between the header box and the top of the text

\usepackage{gvv-book}
\usepackage{gvv}
\usepackage{cite}
\usepackage{amsmath,amssymb,amsfonts,amsthm}
\usepackage{algorithmic}
\usepackage{graphicx}
\usepackage{float}
\usepackage{textcomp}
\usepackage{xcolor}
\usepackage{txfonts}
\usepackage{listings}
\usepackage{enumitem}
\usepackage{mathtools}
\usepackage{gensymb}
\usepackage{comment}
\usepackage[breaklinks=true]{hyperref}
\usepackage{tkz-euclide} 
\usepackage{listings}
% \usepackage{gvv}                                        
\def\inputGnumericTable{}                                 
\usepackage[latin1]{inputenc}                                
\usepackage{color}                                            
\usepackage{array}                                            
\usepackage{longtable}                                       
\usepackage{calc}                                             
\usepackage{multirow}                                         
\usepackage{hhline}                                           
\usepackage{ifthen}                                           
\usepackage{lscape}
\usepackage{tikz}
\usetikzlibrary{patterns}

\begin{document}

\bibliographystyle{IEEEtran}
\vspace{3cm}

\title{12.285}
\author{EE25BTECH11064 - Yojit Manral}

\maketitle
% \maketitle
% \newpage
% \bigskip
{\let\newpage\relax\maketitle}
\renewcommand{\thefigure}{\theenumi}
\renewcommand{\thetable}{\theenumi}
\setlength{\intextsep}{10pt} % Space between text and float

\textbf{Question:}\\
Let $\alpha = e^{2\pi i/5}$ and the matrix
\begin{align}
    \vec{M} = \myvec{1&\alpha&\alpha^2&\alpha^3&\alpha^4\\0&\alpha&\alpha^2&\alpha^3&\alpha^4\\0&0&\alpha^2&\alpha^3&\alpha^4\\0&0&0&\alpha^3&\alpha^4\\0&0&0&0&\alpha^4}
\end{align}
Then the trace of the matrix $\vec{I} + \vec{M} + \vec{M}^2$ is
\begin{enumerate}[label=(\alph*)]
\begin{multicols}{4}
    \item $5$
    \item $0$
    \item $3$
    \item $-5$
\end{multicols}
\end{enumerate}

\textbf{Solution:}\\
$\rightarrow$ As $\alpha$ is the fifth root of unity, we have
\begin{align}
    \alpha^5 - 1 = 0 \implies 1 + \alpha + \alpha^2 + \alpha^3 + \alpha^4 = 0
\end{align}
$\rightarrow$ According to the properties of trace
\begin{align}
    trace(\vec{A}_{n\times n}) &\triangleq \sum_{i=1}^{n} a_{ii} \\
    trace(\vec{I}+\vec{M}+\vec{M}^2) &= trace(\vec{I})+trace(\vec{M})+trace(\vec{M}^2) \\
    trace(\vec{I}_{5\times5}) &= \sum_{i=1}^{5} 1 = 5
\end{align}
$\rightarrow$ From (1) and (3), we get
\begin{align}
    trace(\vec{M}) = \sum_{i=1}^{5} m_{ii} = 1 + \alpha + \alpha^2 + \alpha^3 + \alpha^4
\end{align}
$\rightarrow$ Since $\vec{M}$ is an upper triangular matrix
\begin{align}
    trace(\vec{M}^2) = \sum_{i=1}^{5} m_{ii}^2 = (1)^2 + (\alpha)^2 + (\alpha^2)^2 + (\alpha^3)^2 + (\alpha^4)^2
\end{align}
$\rightarrow$ On adding (5), (6), and (7), using (2) and (4), we get
\begin{align}
    trace(\vec{I}+\vec{M}+\vec{M}^2) &= 5 + (1 + \alpha + \alpha^2 + \alpha^3 + \alpha^4) + (1 + \alpha^2 + \alpha^4 + \alpha\alpha^5 + \alpha^3\alpha^5) \\
    &= 5 + (0) + (0) = 5
\end{align}
$\rightarrow$ Therefore, (a) 5 is the correct option.
\end{document}
