\documentclass{beamer}
\mode<presentation>
\usepackage{amsmath,amssymb,mathtools}
\usepackage{textcomp}
\usepackage{gensymb}
\usepackage{adjustbox}
\usepackage{subcaption}
\usepackage{enumitem}
\usepackage{multicol}
\usepackage{listings}
\usepackage{url}
\usepackage{graphicx} % <-- needed for images
\def\UrlBreaks{\do\/\do-}

\usetheme{Boadilla}
\usecolortheme{lily}
\setbeamertemplate{footline}{
  \leavevmode%
  \hbox{%
  \begin{beamercolorbox}[wd=\paperwidth,ht=2ex,dp=1ex,right]{author in head/foot}%
    \insertframenumber{} / \inserttotalframenumber\hspace*{2ex}
  \end{beamercolorbox}}%
  \vskip0pt%
}
\setbeamertemplate{navigation symbols}{}

\lstset{
  frame=single,
  breaklines=true,
  columns=fullflexible,
  basicstyle=\ttfamily\tiny   % tiny font so code fits
}

\numberwithin{equation}{section}

% ---- your macros ----
\providecommand{\nCr}[2]{\,^{#1}C_{#2}}
\providecommand{\nPr}[2]{\,^{#1}P_{#2}}
\providecommand{\mbf}{\mathbf}
\providecommand{\pr}[1]{\ensuremath{\Pr\left(#1\right)}}
\providecommand{\qfunc}[1]{\ensuremath{Q\left(#1\right)}}
\providecommand{\sbrak}[1]{\ensuremath{{}\left[#1\right]}}
\providecommand{\lsbrak}[1]{\ensuremath{{}\left[#1\right.}}
\providecommand{\rsbrak}[1]{\ensuremath{\left.#1\right]}}
\providecommand{\brak}[1]{\ensuremath{\left(#1\right)}}
\providecommand{\lbrak}[1]{\ensuremath{\left(#1\right.}}
\providecommand{\rbrak}[1]{\ensuremath{\left.#1\right)}}
\providecommand{\cbrak}[1]{\ensuremath{\left\{#1\right\}}}
\providecommand{\lcbrak}[1]{\ensuremath{\left\{#1\right.}}
\providecommand{\rcbrak}[1]{\ensuremath{\left.#1\right\}}}
\theoremstyle{remark}
\newtheorem{rem}{Remark}
\newcommand{\sgn}{\mathop{\mathrm{sgn}}}
\providecommand{\abs}[1]{\left\vert#1\right\vert}
\providecommand{\res}[1]{\Res\displaylimits_{#1}}
\providecommand{\norm}[1]{\lVert#1\rVert}
\providecommand{\mtx}[1]{\mathbf{#1}}
\providecommand{\mean}[1]{E\left[ #1 \right]}
\providecommand{\fourier}{\overset{\mathcal{F}}{ \rightleftharpoons}}
\providecommand{\system}{\overset{\mathcal{H}}{ \longleftrightarrow}}
\providecommand{\dec}[2]{\ensuremath{\overset{#1}{\underset{#2}{\gtrless}}}}
\newcommand{\myvec}[1]{\ensuremath{\begin{pmatrix}#1\end{pmatrix}}}
\let\vec\mathbf

\title{MatGeo Presentation - Problem 12.77}
\author{EE25BTECH11064 - Yojit Manral}
\date{}

\begin{document}

\frame{\titlepage}
\begin{frame}{Question}
Let $\vec{M}$ be a 2$\times$2 real matrix such that
\begin{align}
    (\vec{I}+\vec{M})^{-1} = \vec{I}-\alpha\vec{M}
\end{align}
where $\alpha$ is a nonzero real number and $\vec{I}$ is the 2$\times$2 identity matrix. If the trace of the matrix $\vec{M}$ is $3$, then the value of $\alpha$ is
\begin{enumerate}[label=(\alph*)]
\begin{multicols}{4}
    \item $\frac{3}{4}$
    \item $\frac{1}{3}$
    \item $\frac{1}{2}$
    \item $\frac{1}{4}$
\end{multicols}
\end{enumerate}
\end{frame}

\begin{frame}{Solution}
$\rightarrow$ Post-multiplying by $(\vec{I}+\vec{M})$ on both sides of (1), we get
\begin{align}
    (\vec{I}+\vec{M})^{-1}(\vec{I}+\vec{M}) &= (\vec{I}-\alpha\vec{M})(\vec{I}+\vec{M}) \\
    \vec{I} &= \vec{I} + \vec{M} -\alpha\vec{M} - \alpha\vec{M}^2 \\
    \alpha\vec{M}^2 - (1-\alpha)\vec{M} &= 0
\end{align}
$\rightarrow$ Since (4) is a degree $2$ equation for a $2\times2$ matrix, we can use the Cayley-Hamilton Theorem to get the characteristic equation for the matrix $\vec{M}$ to be
\begin{align}
    \alpha\lambda^2 - (1-\alpha)\lambda = 0
\end{align}
\end{frame}

\begin{frame}{Solution}
$\rightarrow$ In the characteristic equation for any $2\times2$ matrix $\vec{A}$, we know that
\begin{align}
    \lambda^2 - trace(\vec{A})\lambda + det(\vec{A}) = 0
\end{align}
$\rightarrow$ Thus using (5) and (6), we get
\begin{align}
    trace(\vec{M}) = \frac{1-\alpha}{\alpha} &= 3 \\
    1 - \alpha &= 3\alpha \\
    \alpha &= \frac{1}{4}
\end{align}
$\rightarrow$ Therefore, (d) $\frac{1}{4}$ is the correct option.
\end{frame}
\end{document}
