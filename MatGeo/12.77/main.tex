\let\negmedspace\undefined
\let\negthickspace\undefined
\documentclass[journal]{IEEEtran}
\usepackage[a4paper, margin=10mm, onecolumn]{geometry}
\usepackage{lmodern} % Ensure lmodern is loaded for pdflatex
\usepackage{tfrupee} % Include tfrupee package

\setlength{\headheight}{1cm} % Set the height of the header box
\setlength{\headsep}{0mm}  % Set the distance between the header box and the top of the text

\usepackage{gvv-book}
\usepackage{gvv}
\usepackage{cite}
\usepackage{amsmath,amssymb,amsfonts,amsthm}
\usepackage{algorithmic}
\usepackage{graphicx}
\usepackage{float}
\usepackage{textcomp}
\usepackage{xcolor}
\usepackage{txfonts}
\usepackage{listings}
\usepackage{enumitem}
\usepackage{mathtools}
\usepackage{gensymb}
\usepackage{comment}
\usepackage[breaklinks=true]{hyperref}
\usepackage{tkz-euclide} 
\usepackage{listings}
% \usepackage{gvv}                                        
\def\inputGnumericTable{}                                 
\usepackage[latin1]{inputenc}                                
\usepackage{color}                                            
\usepackage{array}                                            
\usepackage{longtable}                                       
\usepackage{calc}                                             
\usepackage{multirow}                                         
\usepackage{hhline}                                           
\usepackage{ifthen}                                           
\usepackage{lscape}
\usepackage{tikz}
\usetikzlibrary{patterns}

\begin{document}

\bibliographystyle{IEEEtran}
\vspace{3cm}

\title{12.77}
\author{EE25BTECH11064 - Yojit Manral}

\maketitle
% \maketitle
% \newpage
% \bigskip
{\let\newpage\relax\maketitle}
\renewcommand{\thefigure}{\theenumi}
\renewcommand{\thetable}{\theenumi}
\setlength{\intextsep}{10pt} % Space between text and float

\textbf{Question:}\\
Let $\vec{M}$ be a 2$\times$2 real matrix such that
\begin{align}
    (\vec{I}+\vec{M})^{-1} = \vec{I}-\alpha\vec{M}
\end{align}
where $\alpha$ is a nonzero real number and $\vec{I}$ is the 2$\times$2 identity matrix. If the trace of the matrix $\vec{M}$ is $3$, then the value of $\alpha$ is
\begin{enumerate}[label=(\alph*)]
\begin{multicols}{4}
    \item $\frac{3}{4}$
    \item $\frac{1}{3}$
    \item $\frac{1}{2}$
    \item $\frac{1}{4}$
\end{multicols}
\end{enumerate}

\textbf{Solution:}\\
$\rightarrow$ Post-multiplying by $(\vec{I}+\vec{M})$ on both sides of (1), we get
\begin{align}
    (\vec{I}+\vec{M})^{-1}(\vec{I}+\vec{M}) &= (\vec{I}-\alpha\vec{M})(\vec{I}+\vec{M}) \\
    \vec{I} &= \vec{I} + \vec{M} -\alpha\vec{M} - \alpha\vec{M}^2 \\
    \alpha\vec{M}^2 - (1-\alpha)\vec{M} &= 0
\end{align}
$\rightarrow$ Since (4) is a degree $2$ equation for a $2\times2$ matrix, we can use the Cayley-Hamilton Theorem to get the characteristic equation for the matrix $\vec{M}$ to be
\begin{align}
    \alpha\lambda^2 - (1-\alpha)\lambda = 0
\end{align}
$\rightarrow$ In the characteristic equation for any $2\times2$ matrix $\vec{A}$, we know that
\begin{align}
    \lambda^2 - trace(\vec{A})\lambda + det(\vec{A}) = 0
\end{align}
$\rightarrow$ Thus using (5) and (6), we get
\begin{align}
    trace(\vec{M}) = \frac{1-\alpha}{\alpha} &= 3 \\
    1 - \alpha &= 3\alpha \\
    \alpha &= \frac{1}{4}
\end{align}
$\rightarrow$ Therefore, (d) $\frac{1}{4}$ is the correct option.
\end{document}
