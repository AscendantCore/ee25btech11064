\let\negmedspace\undefined
\let\negthickspace\undefined
\documentclass[journal]{IEEEtran}
\usepackage[a4paper, margin=10mm, onecolumn]{geometry}
\usepackage{lmodern} % Ensure lmodern is loaded for pdflatex
\usepackage{tfrupee} % Include tfrupee package

\setlength{\headheight}{1cm} % Set the height of the header box
\setlength{\headsep}{0mm}  % Set the distance between the header box and the top of the text

\usepackage{gvv-book}
\usepackage{gvv}
\usepackage{cite}
\usepackage{amsmath,amssymb,amsfonts,amsthm}
\usepackage{algorithmic}
\usepackage{graphicx}
\usepackage{float}
\usepackage{textcomp}
\usepackage{xcolor}
\usepackage{txfonts}
\usepackage{listings}
\usepackage{enumitem}
\usepackage{mathtools}
\usepackage{gensymb}
\usepackage{comment}
\usepackage[breaklinks=true]{hyperref}
\usepackage{tkz-euclide} 
\usepackage{listings}
% \usepackage{gvv}                                        
\def\inputGnumericTable{}                                 
\usepackage[latin1]{inputenc}                                
\usepackage{color}                                            
\usepackage{array}                                            
\usepackage{longtable}                                       
\usepackage{calc}                                             
\usepackage{multirow}                                         
\usepackage{hhline}                                           
\usepackage{ifthen}                                           
\usepackage{lscape}
\usepackage{tikz}
\usetikzlibrary{patterns}

\begin{document}

\bibliographystyle{IEEEtran}
\vspace{3cm}

\title{12.597}
\author{EE25BTECH11064 - Yojit Manral}

\maketitle
% \maketitle
% \newpage
% \bigskip
{\let\newpage\relax\maketitle}
\renewcommand{\thefigure}{\theenumi}
\renewcommand{\thetable}{\theenumi}
\setlength{\intextsep}{10pt} % Space between text and float

\textbf{Question:}\\
The value of p such that the vector $\myvec{1\\2\\3}$ is an eigenvector of the matrix $\myvec{4&1&2\\p&2&1\\14&-4&10}$ is \rule{0.5cm}{0.15mm} .

\textbf{Solution:}\\
$\rightarrow$ If the vector is an eigenvector for the matrix, it satisfies
\begin{align}
    \vec{A}\vec{x} &= \lambda\vec{x} \\
    \myvec{4&1&2\\p&2&1\\14&-4&10} \myvec{1\\2\\3} &= \lambda \myvec{1\\2\\3} \hspace{0.2cm} \exists \lambda \in R \\
    \myvec{12\\p+7\\36} &= \lambda \myvec{1\\2\\3}
\end{align}
$\rightarrow$ Putting $\lambda = 12$ to satisfy (3), we get
\begin{align}
    \myvec{12\\p+7\\36} &= \myvec{12\\24\\36} \\
    \implies p + 7 &= 24 \\
    \implies p &= 15
\end{align}
\end{document}
