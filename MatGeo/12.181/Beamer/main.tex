\documentclass{beamer}
\mode<presentation>
\usepackage{amsmath,amssymb,mathtools}
\usepackage{textcomp}
\usepackage{gensymb}
\usepackage{adjustbox}
\usepackage{subcaption}
\usepackage{enumitem}
\usepackage{multicol}
\usepackage{listings}
\usepackage{url}
\usepackage{graphicx} % <-- needed for images
\def\UrlBreaks{\do\/\do-}

\usetheme{Boadilla}
\usecolortheme{lily}
\setbeamertemplate{footline}{
  \leavevmode%
  \hbox{%
  \begin{beamercolorbox}[wd=\paperwidth,ht=2ex,dp=1ex,right]{author in head/foot}%
    \insertframenumber{} / \inserttotalframenumber\hspace*{2ex}
  \end{beamercolorbox}}%
  \vskip0pt%
}
\setbeamertemplate{navigation symbols}{}

\lstset{
  frame=single,
  breaklines=true,
  columns=fullflexible,
  basicstyle=\ttfamily\tiny   % tiny font so code fits
}

\numberwithin{equation}{section}

% ---- your macros ----
\providecommand{\nCr}[2]{\,^{#1}C_{#2}}
\providecommand{\nPr}[2]{\,^{#1}P_{#2}}
\providecommand{\mbf}{\mathbf}
\providecommand{\pr}[1]{\ensuremath{\Pr\left(#1\right)}}
\providecommand{\qfunc}[1]{\ensuremath{Q\left(#1\right)}}
\providecommand{\sbrak}[1]{\ensuremath{{}\left[#1\right]}}
\providecommand{\lsbrak}[1]{\ensuremath{{}\left[#1\right.}}
\providecommand{\rsbrak}[1]{\ensuremath{\left.#1\right]}}
\providecommand{\brak}[1]{\ensuremath{\left(#1\right)}}
\providecommand{\lbrak}[1]{\ensuremath{\left(#1\right.}}
\providecommand{\rbrak}[1]{\ensuremath{\left.#1\right)}}
\providecommand{\cbrak}[1]{\ensuremath{\left\{#1\right\}}}
\providecommand{\lcbrak}[1]{\ensuremath{\left\{#1\right.}}
\providecommand{\rcbrak}[1]{\ensuremath{\left.#1\right\}}}
\theoremstyle{remark}
\newtheorem{rem}{Remark}
\newcommand{\sgn}{\mathop{\mathrm{sgn}}}
\providecommand{\abs}[1]{\left\vert#1\right\vert}
\providecommand{\res}[1]{\Res\displaylimits_{#1}}
\providecommand{\norm}[1]{\lVert#1\rVert}
\providecommand{\mtx}[1]{\mathbf{#1}}
\providecommand{\mean}[1]{E\left[ #1 \right]}
\providecommand{\fourier}{\overset{\mathcal{F}}{ \rightleftharpoons}}
\providecommand{\system}{\overset{\mathcal{H}}{ \longleftrightarrow}}
\providecommand{\dec}[2]{\ensuremath{\overset{#1}{\underset{#2}{\gtrless}}}}
\newcommand{\myvec}[1]{\ensuremath{\begin{pmatrix}#1\end{pmatrix}}}
\let\vec\mathbf

\title{MatGeo Presentation - Problem 12.181}
\author{EE25BTECH11064 - Yojit Manral}
\date{}

\begin{document}

\frame{\titlepage}
\begin{frame}{Question}
Consider the matrix $\vec{P} = \myvec{0&1\\-2&-3}$. The value of $e^{\vec{P}}$ is
\begin{enumerate}[label=(\alph*)]
\begin{multicols}{2}
    \item $\myvec{2e^{-2}-3e^{-1}&e^{-1}-e^{-2}\\2e^{-2}-2e^{-1}&5e^{-2}-e^{-1}}$
    \item $\myvec{e^{-1}+e^{-2}&2e^{-2}-e^{-1}\\2e^{-1}-4e^{-2}&3e^{-1}+e^{-2}}$
    \item $\myvec{5e^{-2}-6e^{-1}&3e^{-1}-e^{-2}\\2e^{-2}-6e^{-1}&4e^{-2}-e^{-1}}$
    \item $\myvec{2e^{-1}-e^{-2}&e^{-1}-e^{-2}\\-2e^{-1}+2e^{-2}&-e^{-1}+2e^{-2}}$
\end{multicols}
\end{enumerate}
\end{frame}

\begin{frame}{Solution}
$\rightarrow$ To find $e^{\vec{P}}$, we first need to diagonalize the matrix $\vec{P}$ by finding the eigenvalues and their corresponding eigenvectors. So, we use
\begin{align}
    |\vec{P} - \lambda\vec{I}| &= 0 \\
    \left|\begin{array}{cc}-\lambda&1\\-2&-3-\lambda\end{array}\right| &= 0 \\
    \lambda^2 + 3\lambda + 2 &= 0 \\
    \implies \lambda_1 = -1\text{ and }\lambda_2 &= -2
\end{align}
$\rightarrow$ To find the eigenvector for $\lambda_1$, we use the augmented matrix
\begin{align}
    \left(\begin{array}{cc|c}1&1&0\\-2&-2&0\end{array}\right) \xrightarrow{R_2 \leftrightarrow R_2 + 2R_1} \left(\begin{array}{cc|c}1&1&0\\0&0&0\end{array}\right) \\
    \implies \vec{v_1} = \myvec{1\\-1}
\end{align}
\end{frame}

\begin{frame}{Solution}
$\rightarrow$ To find the eigenvector for $\lambda_2$, we use the augmented matrix
\begin{align}
    \left(\begin{array}{cc|c}2&1&0\\-2&-1&0\end{array}\right) \xrightarrow{R_2 \leftrightarrow R_2 + R_1} \left(\begin{array}{cc|c}2&1&0\\0&0&0\end{array}\right) \\
    \implies \vec{v_2} = \myvec{1\\-2}
\end{align}
$\rightarrow$ Thus, after diagonalization, we get
\begin{align}
    \vec{P} &= \vec{Q}\vec{D}\vec{Q}^{-1} \\
    \vec{Q} &= \myvec{\vec{v_1}&\vec{v_2}} = \myvec{1&1\\-1&-2} \\
    \vec{D} &= \myvec{\lambda_1&0\\0&\lambda_2} = \myvec{-1&0\\0&-2} \\
    \vec{Q}^{-1} &= \myvec{2&1\\-1&-1}
\end{align}
\end{frame}

\begin{frame}{Solution}
$\rightarrow$ Now, using property of matrix exponential
\begin{align}
    e^{\vec{P}} &= \vec{Q}e^{\vec{D}}\vec{Q}^{-1} \\
    &= \myvec{1&1\\-1&-2}\myvec{e^{-1}&0\\0&e^{-2}}\myvec{2&1\\-1&-1} \\
    &= \myvec{e^{-1}&e^{-2}\\-e^{-1}&-2e^{-2}}\myvec{2&1\\-1&-1} \\
    e^{\vec{P}} &= \myvec{2e^{-1}-e^{-2}&e^{-1}-e^{-2}\\-2e^{-1}+2e^{-2}&-e^{-1}+2e^{-2}}
\end{align}
$\rightarrow$ Therefore, (d) $\myvec{2e^{-1}-e^{-2}&e^{-1}-e^{-2}\\-2e^{-1}+2e^{-2}&-e^{-1}+2e^{-2}}$ is the correct option.
\end{frame}
\end{document}
