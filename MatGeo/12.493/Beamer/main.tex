\documentclass{beamer}
\mode<presentation>
\usepackage{amsmath,amssymb,mathtools}
\usepackage{textcomp}
\usepackage{gensymb}
\usepackage{adjustbox}
\usepackage{subcaption}
\usepackage{enumitem}
\usepackage{multicol}
\usepackage{listings}
\usepackage{url}
\usepackage{graphicx} % <-- needed for images
\def\UrlBreaks{\do\/\do-}

\usetheme{Boadilla}
\usecolortheme{lily}
\setbeamertemplate{footline}{
  \leavevmode%
  \hbox{%
  \begin{beamercolorbox}[wd=\paperwidth,ht=2ex,dp=1ex,right]{author in head/foot}%
    \insertframenumber{} / \inserttotalframenumber\hspace*{2ex}
  \end{beamercolorbox}}%
  \vskip0pt%
}
\setbeamertemplate{navigation symbols}{}

\lstset{
  frame=single,
  breaklines=true,
  columns=fullflexible,
  basicstyle=\ttfamily\tiny   % tiny font so code fits
}

\numberwithin{equation}{section}

% ---- your macros ----
\providecommand{\nCr}[2]{\,^{#1}C_{#2}}
\providecommand{\nPr}[2]{\,^{#1}P_{#2}}
\providecommand{\mbf}{\mathbf}
\providecommand{\pr}[1]{\ensuremath{\Pr\left(#1\right)}}
\providecommand{\qfunc}[1]{\ensuremath{Q\left(#1\right)}}
\providecommand{\sbrak}[1]{\ensuremath{{}\left[#1\right]}}
\providecommand{\lsbrak}[1]{\ensuremath{{}\left[#1\right.}}
\providecommand{\rsbrak}[1]{\ensuremath{\left.#1\right]}}
\providecommand{\brak}[1]{\ensuremath{\left(#1\right)}}
\providecommand{\lbrak}[1]{\ensuremath{\left(#1\right.}}
\providecommand{\rbrak}[1]{\ensuremath{\left.#1\right)}}
\providecommand{\cbrak}[1]{\ensuremath{\left\{#1\right\}}}
\providecommand{\lcbrak}[1]{\ensuremath{\left\{#1\right.}}
\providecommand{\rcbrak}[1]{\ensuremath{\left.#1\right\}}}
\theoremstyle{remark}
\newtheorem{rem}{Remark}
\newcommand{\sgn}{\mathop{\mathrm{sgn}}}
\providecommand{\abs}[1]{\left\vert#1\right\vert}
\providecommand{\res}[1]{\Res\displaylimits_{#1}}
\providecommand{\norm}[1]{\lVert#1\rVert}
\providecommand{\mtx}[1]{\mathbf{#1}}
\providecommand{\mean}[1]{E\left[ #1 \right]}
\providecommand{\fourier}{\overset{\mathcal{F}}{ \rightleftharpoons}}
\providecommand{\system}{\overset{\mathcal{H}}{ \longleftrightarrow}}
\providecommand{\dec}[2]{\ensuremath{\overset{#1}{\underset{#2}{\gtrless}}}}
\newcommand{\myvec}[1]{\ensuremath{\begin{pmatrix}#1\end{pmatrix}}}
\let\vec\mathbf

\title{MatGeo Presentation - Problem 12.493}
\author{EE25BTECH11064 - Yojit Manral}
\date{}

\begin{document}

\frame{\titlepage}
\begin{frame}{Question}
Characteristic equation of the matrix with eigenvalue $\lambda$ is
\begin{align}
    \vec{A} = \myvec{2&\sqrt{2}\\ \sqrt{2}&1}
\end{align}
\begin{enumerate}[label=(\alph*)]
\begin{multicols}{2}
    \item $\lambda^2+3\lambda+4=0$
    \item $\lambda^2+3\lambda-2=0$
    \item $\lambda^2-3\lambda=0$
    \item $\lambda^2+3\lambda=0$
\end{multicols}
\end{enumerate}
\end{frame}

\begin{frame}{Solution}
$\longrightarrow$ This problem statement can be solved via the following two methods:
\newline
\begin{enumerate}[label=\arabic*)]
    \item {
    The characteristic equation can be given by
    \begin{align}
        char(\vec{A}) \implies |\vec{A}-\lambda\vec{I}| &= 0 \\
        \left|\begin{array}{cc}2-\lambda&\sqrt{2}\\ \sqrt{2}&1-\lambda\end{array}\right| &= 0 \\
        (\lambda-2)(\lambda-1)-2 &= 0 \\
        \lambda^2-3\lambda &= 0
    \end{align}
    }
\end{enumerate}
\end{frame}

\begin{frame}{Solution}
\begin{enumerate}[label=\arabic*)]\setcounter{enumi}{1}
    \item {
    Another method to find characteristic equation for a $2\times2$ matrix is
    \begin{align}
        char(\vec{M}_{2\times2}) \implies \lambda^2-trace(\vec{M})\lambda+det(\vec{M}) &= 0
    \end{align}
    For the given matrix $\vec{A}$, we have
    \begin{align} trace(\vec{A}) = 3 && det(\vec{A}) = 0 \end{align}
    From (6) and (7), we get
    \begin{align}
        char(\vec{A}) \implies \lambda^2-3\lambda = 0
    \end{align}
    }
\end{enumerate}
$\longrightarrow$ Therefore, \textit{(c) $\lambda^2-3\lambda=0$} is the correct option.
\end{frame}
\end{document}
